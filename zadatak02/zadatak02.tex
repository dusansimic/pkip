\documentclass[11pt]{article}
\usepackage[utf8]{inputenc}
\usepackage[serbian]{babel}
\usepackage{amsmath, amssymb, amsthm}
\usepackage{indentfirst}

% Custom theorem styles to match the ones from professor
\newtheoremstyle{masulthm}%
{\parsep}%
{\parsep}%
{\itshape}%
{}%
{\bfseries}%
{}%
{.5em}%
{\thmname{#1}\thmnumber{ #2}\thmnote{ (#3)}}%

\newtheoremstyle{masuldef}%
{\parsep}%
{\parsep}%
{\itshape}%
{}%
{\bfseries}%
{}%
{.5em}%
{\thmname{#1}\thmnumber{ #2}\thmnote{ (#3)}}%

\newtheoremstyle{masulexmp}%
{\parsep}%
{\parsep}%
{}%
{}%
{\bfseries}%
{.}%
{.5em}%
{\thmname{#1}\thmnumber{ #2}\thmnote{ (#3)}}%

\newtheoremstyle{masulproof}{\topsep}{\topsep}{}{}{}{.}{.5em}{\thmname{#1}}


\theoremstyle{masulthm}
\newtheorem{theorem}{Teorema}[section]
\theoremstyle{masuldef}
\newtheorem{definition}{Definicija}[section]
\theoremstyle{masulexmp}
\newtheorem*{exmp*}{Primer}
\theoremstyle{masulproof}
\newtheorem*{masulproof}{Dokaz}

\newcommand{\inv}[1]{\text{inv}(#1)}
\newcommand{\R}{\mathbb{R}}

\title{Kratak uvod u matrice i determinante}
\author{Dragan Mašulović}
\date{7. maj 2020.}

\begin{document}

\maketitle

\begin{abstract}

Analiziranje rešenja sistema jednačina nas vodi do pojma determinante, čemo je posvećen početak
poglavlja. Nakon toga uvodimo pojam matrice i pokazujemo da je najsločenija i najneintuitivnija
operacija nad matricama, množenje matrica, zapravo posledica analize ponašanja linearnih transformacija.
Rešavanje sistema linearnih jednačina pokazujemo koristeći nekoliko strategija, od kojih neke imaju
samo teorijski značaj, i glavu zaključujemo demonstracijom tehnike za nalaženje inverzne matrice date matrice.
    
\end{abstract}

\section{Rešavanje sistema jednačina}

Rešavanje sistema linearnih jednačina predstavlja važnu
temu u savremenoj matematici jer se mnogi fenomeni u nauci mogu modelovati
sistemima linearnih jednačina. Tipičan primer predstavlja analiza
električnih kola primenom Kirhofovih zakona \cite{boyd}.
Drugi važan izvor motivacije za razmatranja u ovom poglavlju su linearne transformacije \cite{axler}.
Recimo, Lorencova transformacija opisuje promenu parametara (položaja i vremena) materijalne tačke
u jednom referentnom sistemu kada se ona posmatra iz drugog referentnog sistema. Ako pretpostavimo
da se referentni sistem u kome se nalazi materijalna tačka kreće konstantnom brzinom $ v $, pravolinijski
duž $ x $-ose referentnog sistema posmatrača, onda se veza izme\dj u koordinata
$ (x, y, z, t) $ materijalne tačke
u njenom referentnom sistemu i koordinata $ (x', y', z', t') $ u referentnom sistemu posmatrača
može opisati na sledeći način:

\begin{align*}
    x' & = \gamma (x - vt) \\
    y' & = y \\
    z' & = z \\
    t' & = \frac{\gamma}{c^2}(c^2 t - vx)
\end{align*}

\noindent
gde je $ c $ brzina svetlosti i $ \gamma = \frac{1}{\sqrt{1 - \frac{v^2}{c^2}}} $. Primetimo da su
$ c $, $ v $ i $ \gamma $
konstante. Ovo je primer transformacije koordinata $ (x, y, z, t) \mapsto (x', y', z' t') $.
Transformacija koordinata ne mora da preslikava neke koordinate na isti broj koordinata. U računarskoj
grafici se sistematski koriste transformacije $ (x, y, z) \mapsto (x' , y') $ koje koordinate iz trodimenzionalnog
prostora (koji nas okružuje) preslikavaju na dve dimenzije (ekrana računara). Na primer, kosa projekcija
se može opisati sledećom transformacijom koordinata:
\begin{align*}
    x' & = x - \frac{\sqrt{2}}{8} z \\
    y' & = y - \frac{\sqrt{2}}{8} z.
\end{align*}

\noindent
I Lorencove transformacije i kosa projekcija predstavljaju samo specijalan slučaj opšte linearne tranformacije $ (x_1, x_2, \dots, x_n) \mapsto (y_1, y_2, \dots, y_m) $:
\begin{align*}
    y_1 & = a_{11} x_1 + a_{12} x_2 + \dots + a_{1n} x_n \\
    y_2 & = a_{21} x_1 + a_{22} x_2 + \dots + a_{2n} x_n \\
    \vdots \\
    y_m & = a_{m1} x_1 + a_{m2} x_2 + \dots + a_{mn} x_n
\end{align*}

\noindent
gde su $ a_{ij} $ neki realni brojevi.

\section{Definicija determinante}

\begin{definition}

Neka je $ n $ prirodan broj i neka su $ a_{ij} $ proizvoljni realni brojevi, $ 1 \leq i, j \leq n $.
Determinanta reda $ n $ je algebarski izraz oblika
\begin{align*}
    D =
    \begin{vmatrix}
        a_{11} & a_{12} & \dots & a_{1n} \\
        a_{21} & a_{22} & \dots & a_{2n} \\
        \vdots & \vdots &       & \vdots \\
        a_{n1} & a_{n2} & \dots & a_{nn}
    \end{vmatrix}
    = \sum_{f \in S_{n}} (-1)^{\inv{f}} a_{1f_1} a_{2f_2} \dots a_{nf_n}
\end{align*}

gde je $ S_n $ skup svih permutacija skupa $ {1, 2, \dots, n} $.
Kraće pišemo i $ D = |a_{ij}|_{n \times n} $.

\end{definition}

\noindent
Tri kratka komentara:

\begin{itemize}
    \item kao što smo i nagovestili, sumiranje se vrši po svim permutacijama skupa $ \{1, 2, \dots, n\} $;

    \item zato je determinanta reda $ n $ algebarski izraz koji ima $ n! $ sabiraka;

    \item ako je $ f $ parna permutacija onda je broj $ \inv{f} $ paran, pa je $ (-1)^{\inv{f}} = 1 $;
ako je $ f $ neparna permutacija onda je broj $ \inv{f} $ neparan, pa je $ (-1)^{\inv{f}} = -1 $;
tako smo zapisali činjenicu da se sabirci koji odgovaraju parnim permutacijama uzimaju sa znakom $ + $,
dok se sabirci koji odgovaraju neparnim permutacijama uzimaju sa znakom $ - $.
\end{itemize}



\section{Osobine determinanti}

\begin{theorem}[Vrsta/kolona koja se sastoji od nula]
Ako se u determinanti $ D $ neka vrsta ili kolona sastoji isključivo od nula, onda je $ D = 0 $.
\end{theorem}

\noindent
\begin{masulproof}
    Primetimo, prvo, da svaki sabirak u definiciji determinante sadrži tačno jedan element svake
    vrste determinante, i tačno jedan element svake kolone determinante.
    To je ključni sastojak u dokazu koji sledi.
    Neka se u determinanti $ D $ vrsta $ i $ sastoji isključivo od nula:
    $ a_{i1} = a_{i2} = \dots = a_{in} = 0 $.
    Tada je $ a_{if_i} = 0 $ za svako $ f \in S_n $ što znači da je svaki sabirak u izrazu%
    \begin{equation*}
        \sum_{f \in S_n} (-1)^{\inv{f}} a_{1f_1} \dots a_{if_i} \dots a_{nf_n}
    \end{equation*}
    jednak nuli. Dakle, $ D = 0 $.
\end{masulproof}

Osim ove osobine, determinante imaju još niz važnih osobina koje koristimo da bismo
efektivno odrediti vrednost determinante. Sada samo navodimo osobine i tek po koji dokaz, dok ostale, često veoma komplikovane
dokaze dajemo u sledećem odeljku za posebno motivisane čitaoce.

\begin{theorem}[Vrste postaju kolone i obrnuto]

Determinanta ne menja vrednost ukoliko njene vrste postanu kolone tako da prva vrsta postane
prva kolona, druga vrsta postane druga kolona, \dots, poslednja vrsta postane poslednja kolona. Preciznije,
\newcommand{\D}{\vphantom{\vdots}}
\begin{align*}
    \begin{vmatrix}
        a_{11} & a_{12} & \dots & a_{1n} \D \\
        a_{21} & a_{22} & \dots & a_{2n} \D \\
        \dots  & \dots  &       & \dots  \D \\
        a_{n1} & a_{n2} & \dots & a_{nn} \D \\
    \end{vmatrix}
    =
    \begin{vmatrix}
        a_{11} & a_{21} & \vdots & a_{n1} \\
        a_{12} & a_{22} & \vdots & a_{n2} \\
        \vdots & \vdots &        & \vdots \\
        a_{1n} & a_{2n} & \vdots & a_{nn} \\
    \end{vmatrix}.
\end{align*}

\end{theorem}

Prethodna osobina determinanti je važna zato što pokazuje da ako neka osobina determinanti važi za vrste
determinante, važi i za kolone determinante. Ovaj neformalni princip ćemo ilustrovati na brojnim primerima koji slede.

\begin{theorem}[\mbox{Zamena mesta dvema vrstama/kolonama determinante}]
Neka je $ D' $ determinanta koja se dobija od determinante $ D $ tako što dve vrste (ili dve kolone)
zamene mesta. Tada je $ D' = -D $.
\end{theorem}

\begin{theorem}[Dve jednake vrste/kolone]
Ako su u determinanti $ D $ neke dve vrste jednake, ili neke dve kolone jednake, onda je $ D = 0 $.
\end{theorem}



\section{Kramerova teorema}

Kramerova teorema predstavlja važan teorijski rezultat koji nam opisuje strukturu rešenja
kvadratnih sistema linearnih jednačina. Ova teorema nije pogodna za efikasno rešavanje sistema
linearnih jednačina i zato je njen značaj isključivo teorijski.
Posmatrajmo sledeći sistem $ n $ linearnih jednačina sa $ n $ nepoznatih:
\begin{align*}
    a_{11} x_1 + a_{12} x_2 + \dots + a_{1n} x_n & = b_1  \\
    a_{21} x_1 + a_{22} x_2 + \dots + a_{2n} x_n & = b_2  \\
    & \mathrel{
        \settowidth{\dimen0}{$=$}
        \hbox to \dimen0{\hss$\vdots$\hss}
    } \\
    a_{n1} x_1 + a_{n2} x_2 + \dots + a_{nn} x_n & = b_n.
\end{align*}

\noindent
Neka je $ D = |a_{ij}|_{n \times n} $ determinanta sistema i neka je, za svako $ i \in \{1, \dots, n\}$,
sa $ D_x $ označena determinanta koja se dobija tako što se $ i $-ta kolona
determinante $ D $ zameni kolonom koju čine slobodni članovi:

\begin{equation*}
    D_{x_1} =
    \begin{vmatrix}
        a_{11} & \dots & a_{1,i-1} & b_1 & a_{1,i+1} & \dots & a_{1n} \\
        a_{21} & \dots & a_{2,i-1} & b_2 & a_{2,i+1} & \dots & a_{2n} \\
        \vdots &       & \vdots    & \vdots & \vdots &       & \vdots \\
        a_{n1} & \dots & a_{n,i-1} & b_n & a_{n,i+1} & \dots & a_{nn} \\
    \end{vmatrix}
\end{equation*}

\begin{theorem}[Kramerova teorema]
Ako za kvadratni sistem reda $ n $ znamo da je $ D \ne 0 $,
tada je sistem odre\dj en i njegovo jedinstveno rešenje je dato sa:
\begin{equation*}
    x_1 = \frac{D_{x_1}}{D},
    x_2 = \frac{D_{x_2}}{D},
    \dots,
    x_n = \frac{D_{x_n}}{D}.
\end{equation*}

\end{theorem}

\noindent
\begin{masulproof}
  Neka je $ D_{(pq)} $ determinanta reda $ n - 1 $ koja se dobija izbacivanjem
  $ p $-te vrste i $ q $-te kolone determinante $ D $.
  Kada prvu jednačinu sistema pomnožimo sa $ D_{(11)} $, drugu pomnožimo sa
  $ -D_{(21)} $, treću sa $ D_{(31)} $, \dots, $ i $-tu sa
  $ (-1)^{i+1} D_{(i1)} $, i sve to saberemo, dobijamo:
  \begin{align*}
      & (\underbrace{a_{11} D_{(11)} - a_{21} D_{(21)} + \ldots + (-1)^{n+1} a_{n1} D_{(n1)}}_D) x_1 + \\
      & (\underbrace{a_{12} D_{(11)} - a_{22} D_{(21)} + \ldots + (-1)^{n+1} a_{n2} D_{(n1)}}_0) x_2 + \ldots + \\
      & (\underbrace{a_{1n} D_{(11)} - a_{2n} D_{(21)} + \ldots + (-1)^{n+1} a_{nn} D_{(n1)}}_0) x_n = \\
      & = \underbrace{b_1 D_{(11)} - b_2 D_{(21)} + \ldots + (-1)^{n+1} b_n D_{(n1)}}_{D_{x_1}} .
  \end{align*}
  
  \noindent
  Izraz koji moži promenljivu $ x_1 $ predstavlja razvoj determinante $ D $
  po elementima njene prve kolone, dok izrazi koji množe promenljive $ x_2 $, \dots, $ x_n $ predstavljaju razvoj determinante $ D $ po elementima ``pogrešne'' kolone.
  S druge strane, slobodni član predstavlja razvoj determinante $ D_{x_1} $ po elementima njene prve kolone.
  Koristeći teoremu o razvoju determinante po elementima prve kolone, i koristeći činjenicu da je
  razvoj determinante po elementima ``pogrešne kolone'' jednak 0, prethodna jednačina se može
  zapisati ovako:
  \begin{equation*}
      D \cdot x_1 = D_{x_1}.
  \end{equation*}
  
  \noindent
  Kako je $ D \ne 0 $ dobijamo $ x = \frac{D_{x_1}}{D} $. Na isti način se dobija da je
  $ X_i = \frac{D_{x_i}}{D} $ za sve $ i \in \{2, \dots, n\} $.
\end{masulproof}

\section{Pojam matrice, operacije sa matricama}

\emph{Matrica} je pravougaona šema brojeva čije dimenzije zovemo
format matrice \cite{klein}. Na primer:

\begin{tabular}{ccccc}
    Matrica: &
    
    $
    \left[\begin{array}{cccc}
    5 & 3 & 9 & 9 \\
    2 & 0 & 4 & 1 \\
    0 & 0 & 0 & 5 \\
    \end{array}\right]
    $ &
    
    $
    \left[\begin{array}{cccc}
    5 \\ 2 \\ 0 \\ 6
    \end{array}\right]
    $ &
    
    $
    \left[\begin{array}{cccc}
    5 & 3 & 34 & 9
    \end{array}\right]
    $ &
    
    $
    \left[\begin{array}{cccc}
    3 & 9 & 9 \\
    0 & 4 & 1 \\
    0 & 0 & 5
    \end{array}\right]
    $ \\

    Format: &
    $ 3 \times 4 $ &
    $ 4 \times 1 $ &
    $ 1 \times 4 $ &
    $ 3 \times 3 $
\end{tabular}

\noindent
Matrica formata $ k \times k $ zove se \emph{kvadratna matrica}.
Matrica formata $ 1 \times k $ zove se \emph{vektor-vrsta},
a matrica formata $ k \times 1 $ \emph{vektor-kolona}. Zapis
$ A = [a_{ij}]_{n \times m} $ znači da je $ A $ matrica formata
$ n \times m $ čiji elementi su označeni sa $ a_{ij} $. Pri
tome je $ i \in \{1, \dots, n\} $, a $ j \in \{1, \dots, m\} $.

Skup svih matrica formata $ m \times n $ sa elementima iz $ \R $
označavamo sa $ \R^{m \times n} $. Matrice istog formata se sabiraju tako što se saberu elementi na odgovarajućim mestima:

\begin{center}
    \begin{tabular}{c}
            $\begin{bmatrix}
                a_{11} & a_{12} & \dots & a_{1n} \\
                a_{21} & a_{22} & \dots & a_{2n} \\
                \vdots & \vdots &       & \vdots \\
                a_{m1} & a_{m2} & \dots & a_{mn}
            \end{bmatrix}
            +
            \begin{bmatrix}
                b_{11} & b_{12} & \dots & b_{1n} \\
                b_{21} & b_{22} & \dots & b_{2n} \\
                \vdots & \vdots &       & \vdots \\
                b_{m1} & b_{m2} & \dots & b_{mn}
            \end{bmatrix}
            = $\\
            $= \begin{bmatrix}
                a_{11} + b_{11} & a_{12} + b_{12} & \dots & a_{1n} + b_{1n} \\
                a_{21} + b_{21} & a_{22} + b_{22} & \dots & a_{2n} + b_{2n} \\
                \vdots & \vdots &       & \vdots \\
                a_{m1} + b_{m1} & a_{m2} + b_{m2} & \dots & a_{mn} + b_{mn}
            \end{bmatrix}.$
    \end{tabular}
\end{center}

\noindent
Kraće pišemo još i ovako:

\begin{equation*}
    [a_{ij}]_{m \times n} + [b_{ij}]_{m \times n} = [a_{ij} + b_{ij}]_{m \times n}.
\end{equation*}

\noindent
Matrica se množi brojem tako što se svaki element matrice pomnoži tim brojem:

\begin{equation*}
    \alpha
    \begin{bmatrix}
        a_{11} & a_{12} & \dots & a_{1n} \\
        a_{21} & a_{22} & \dots & a_{2n} \\
        \vdots & \vdots &       & \vdots \\
        a_{m1} & a_{m2} & \dots & a_{mn}
    \end{bmatrix}
    =
    \begin{bmatrix}
        \alpha a_{11} & \alpha a_{12} & \dots & \alpha a_{1n} \\
        \alpha a_{21} & \alpha a_{22} & \dots & \alpha a_{2n} \\
        \vdots & \vdots &       & \vdots \\
        \alpha a_{m1} & \alpha a_{m2} & \dots & \alpha a_{mn}
    \end{bmatrix}.
\end{equation*}

\noindent
Kraće pišemo još i ovako:

\begin{equation*}
    \alpha \cdot [a_{ij}]_{m \times n} = [\alpha \cdot a_{ij}]_{m \times n}.
\end{equation*}

Sada možemo da pre\dj emo na opštu definiciju proizvoda matrica. Prvo, podsetimo se da 
matrice mogu da se pomnože samo ako je broj kolona prve matrice jednak broju vrsta druge.
Za matrice $ A $ formata $ k \times l $ i $ B $ formata $ m \times n $ kažemo
da su \emph{kompatibilne} ako je $ l = n $. \emph{Proizvod} kompatibilnih
matrica $ A = [a_{ij}]_{k \times n} $ i $ B = [b_{ij}]_{n \times m} $, pišemo $ C = AB $, je
matrica $ C = [c_{ij}]_{k \times m} $ čiji elementi se računaju na
sledeći način:

\begin{equation*}
    c_{ij} = \sum_{s = 1}^n a_{is} b_{sj}.
\end{equation*}

\begin{theorem}
    Moženje matrica je asocijativno. Preciznije, ako je $ A $ matrica formata $ m \times n $,
    $ B $ matrica formata $ n \times p $ i $ C $ matrica formata
    $ p \times q $, onda je $ (AB)C = A(BC) $.
\end{theorem}

Dokaz tvr\dj enja nećemo dati u ovom kursu iako je ideja dokaza u osnovi jednostavna: množenje matrica
odgovara kompoziciji specijalnih preslikavanja koje smo zvali linearne transformacije, pa kako je
kompozicija funkcija asocijativna operacija, to ne čudi da je i množenje matrica asocijativna operacija.

\begin{exmp*}
    Množenje matrica u opštem slučaju nije komutativno. Na primer,
    $
    \begin{bmatrix}
        1 & 2 \\
        3 & 4
    \end{bmatrix}
    \cdot
    \begin{bmatrix}
        1 & 0 \\
        0 & 0
    \end{bmatrix}
    =
    \begin{bmatrix}
        1 & 0 \\
        3 & 0
    \end{bmatrix}
    $, dok je
    $
    \begin{bmatrix}
        1 & 0 \\
        0 & 0
    \end{bmatrix}
    \cdot
    \begin{bmatrix}
        1 & 2 \\
        3 & 4
    \end{bmatrix}
    =
    \begin{bmatrix}
        1 & 2 \\
        0 & 0
    \end{bmatrix}
    $.
    
    Postoji matrica koja prilikom množenja matrica ima istu ulogu koju broj 1 ima prilikom množenja brojeva.
    Sa $ E_n $ označavamo sledeću kvadratnu matricu reda $ n $:
    
    \begin{equation*}
        E_n =
        \begin{bmatrix}
            1 & 0 & 0 & \dots & 0 \\
            0 & 1 & 0 & \dots & 0 \\
            0 & 0 & 1 & \dots & 0 \\
            \vdots & \vdots & \vdots & & \vdots \\
            0 & 0 & 0 & \dots & 1
        \end{bmatrix},
    \end{equation*}
    
    i zovemo je \emph{jedinična matrica reda} $ n $.
    Ukoliko je red matrice $ E_n $ jasan iz konteksta, umesto $ E_n $ pišemo samo $ E $.
\end{exmp*}

\begin{theorem}
  Za proizvoljnu matricu $ A $ formata $ m \times n $ je $ AE_n = E_mA = A $.
\end{theorem}

Svakoj kvadratnoj matrici $ A = [a_{ij}]_{n \times n} $ se prirodno dodeljuje determinanta
$ D = [a_{ij}]_{n \times n} $ reda $ n $. Kažemo da je $ D $ determinanta matrice $ A $.

Naredna teorema pokazuje da se determinanta ``slaže'' sa množenjem kvadratnih matrica istog reda
u sledećem smislu: $ |AB| = |A| \cdot |B| $.

\begin{theorem}
  Neka su $ A $ i $ B $ kvadratne matrice istog reda. Tada je $ |AB| = |A| \cdot |B| $.
\end{theorem}

\begin{thebibliography}{9}
\bibitem{axler}
Sheldon Axler: ``Linear Algebra Done Right'' (2nd Ed.), Springer-Verlag New York Berlin Heidelberg 1997.

\bibitem{boyd}
Stephen Boyd, Lieven Vandenberghe: ``Introduction to Applied Linear Algebra: Vectors, Matrices, and Least Square'',
Cambridge University Press 2018.

\bibitem{klein}
Philip N. Klein: ``Coding the Matrix -- Linear Algebra through Applications to Computer Science''.
Newtonian Press 2015.

\end{thebibliography}

\end{document}
