\documentclass{article}
\usepackage[serbianc]{babel}

\begin{document}
	\section{Prvi zadatak}
	\begin{enumerate}
		\item Recite \v zelju i bi\' ce ma\dj ija d\v zinovskog duha iz fla\v se po\v cinje.
		\item Odbaci\' ce kavgad\v zija pla\v stom \v ca\dj u \v zeljezni fenjer.
		\item \DJ a\v ce, u\v stedu pla\' caj \v zaljenjem zbog d\v zinovskih cifara.
		\item Boja va\v se haljine, gospodice D\v zafi\' c, tra\v zi da za nju kulu\v cim.
	\end{enumerate}

	\section{Drugi zadatak}
	\begin{enumerate}
		\item K\' rdel' \v st'astn\' ych d'atl'ov u\v ci pri \' usti V\' ahu m\' lkveho ko\v na obhr\' yzat' k\^ oru a \v zrat' \v cerstv\' e m\" aso. (slova\v cki)
		\item P\v r\i li\v s \v zlut'ou\v ck\' y k\r u\u n ' up\u el d'\' abelsk\' e \' ody. (\v ce\v ski)
		\item J\' o foxim don Quijote h\' uszwattos l\' amp\' an\' al \" ulve egy p\' ar b\H uv\" os cip\H ot k\' eszit. (ma\dj arski)
		\item H\o vdingens kj\ae re squaw f\r ar litt pizza i Mexico by. (norve\v ski)
		\item Pijamal{\i} hasta ya\u giz \c sof\" ore cabucak g\" uvendi. (turski)
	\end{enumerate}

	\section{Tre\' ci zadatak}
	\textbf{ВАШИНГТОН} - Америчка свемирска агенција (НАСА) приказала је снимке са ровера ``Презерванс'' (Perseverance Mars Rover), 
	који је ушао у атмосферу Марса, спустио се и слетео на црвену планету 18. фебруара, а микрофон је такође емитовао први 
	аудио-снимак звукова са ове планете, саопштила је НАСА.

	Од тренутка када се активирао падобран, систем камера је приказао читав процес спуштања, приказујући динамичну вожњу ровера 
	до кратера ``Језеро'', преноси Танјуг. Снимак започиње 11 километара изнад површине и приказује отварање највећег падобрана 
	икада послатог ван Земље, а завршава се додиром ровера и кратера.

	``За оне који се питају како је спустити се на Марс, зашто је то тако тешко или како би било `кул' да се то уради, морате да 
	гледате мало даље. `Презерванс' се тек загрева и већ нам је послао неке историјске снимке у истраживању свемира. Потврдио је 
	изванредан ниво на којем је изграђен и прецизност која је неопходна да се направи летелица и спусти ровер на Марс'', рекао је 
	вршилац дужности председника НАСА Стив Јурчјак.
\end{document}
